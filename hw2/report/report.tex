\documentclass[12pt]{article}
\usepackage[left=2cm,right=2cm,top=2cm,bottom=2cm,bindingoffset=0cm]{geometry}
\usepackage[utf8x]{inputenc}
\usepackage[russian]{babel}
\usepackage{amsmath}
\usepackage{hyperref}

\title{Задача оптимизации расписания с использованием алгоритма имитации отжига}
\author{Забровский Валерий Владимирович}
\date{Вариант 1}

\begin{document}
\maketitle

\section*{Формальная постановка задачи}

\subsection*{Дано}
\begin{itemize}
    \item множество независимых работ $W = \{W_i\}_{i=1}^{N}$;
    \item время выполнения $t_i$ работы $W_i$, $i = \overline{1, N}$;
    \item множество процессоров $P = \{p_i\}_{i=1}^{M}$.
\end{itemize}

\subsection*{Требуется}
Построить расписание выполнения работ $W$ на $M$ процессорах без прерываний так,
что достигается минимум критерия~\hyperref[crit]{$K_1$}.

\subsection*{Расписание}

Расписание представлено в виде булевой матрицы $S \in \{0, 1\}^{M \times N}$,
причём $S_{ij} = 1$, если на процессоре $p_i$ выполняется работа $W_j$,
и $S_{ij} = 0$ иначе.

Для каждого процессора $p_i$, $i = \overline{1, M}$, введём множество
всех работ, определённых на выполнение на данном процессоре:
\begin{equation}
    \Pi_i = \bigcup\limits_{\substack{j=1 \\ S_{ij}=1}}^N \{W_j\}.
\end{equation}
При этом $\bigcup\limits_{i=1}^M \Pi_i = W$.

Для каждой работы $W_j$, $j = \overline{1, N}$, обозначим время начала и конца
выполнения через $T_B(W_j)$ и $T_E(W_j)$ соответственно.
При этом $0 \leq T_B(W_j) \leq T_E(W_j)$.

\subsection*{Минимизируемый критерий}
\label{crit}
Разбалансированность расписания
\begin{equation}
    K_1 = T_{\max} - T_{\min},
\end{equation}
\begin{equation}
    T_{\max} = \max_{j = \overline{1, N}} T_E(W_j),
\end{equation}
\begin{equation}
    T_{\min} = \min_{j = \overline{1, N}} T_E(W_j).
\end{equation}

\subsection*{Ограничения}
Каждая работа $W_j$, $\forall j = \overline{1, N}$:
\begin{itemize}
    \item выполняется без прерываний, т.е.
    \begin{equation}
        T_E(W_j) = T_B(W_j) + t_j
    \end{equation}
    и
    \begin{equation}
        \forall p_i~(|\Pi_i| > 0)~\exists! W_k \in \Pi_i: T_B(W_k) = 0,
    \end{equation}

    \item выполняется строго на одном процессоре, т.е.
    $\exists! p_i \in P: S_{ij} = 1;$

    \item имеет фиксированное время выполнения: $t_j = \text{const}$.
\end{itemize}

\end{document}
